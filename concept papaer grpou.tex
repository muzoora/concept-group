\documentclass {article}
\pagenumbering {arabic}
\begin{document}
\title {BETTER PERFORMANCE OF GEOMLAB}
\author {AMPAIRE WILFUL 15/U/21173}
\maketitle
\section {Introduction}
GeomLab is a tiny programming environment for a simple functional programming language with graphics primitives.
 It is used to introduce high school students to computer programming and some of the most important ideas in computer programming in an interactive
\subsection {Background }
Programming using geomlab requires very many work sheets which are run only when you are connected to the internet .It also requires the use of a specific programming language makes it more complicated. Due to all the challenges being faced, programmers waste a lot of time while working with all the work sheets and searching for internet ,and programmers used to other programming languages are discouraged        


\subsection {Problem statement}
New programmers find it hard to interpret recursion which is a very powerful mathematical concept. More so the software students use require a series of worksheets which is hard for starting programmers. Geomlab has a turtle graphics feature, but the pictures are drawn only on the screen.
\subsection{ Aim and objectives}
\subsubsection{ General objective}
To enhance the performance of geomlab
 \subsubsection {Specific objectives}
To establish a system that can run and be accessed on all kinds of computers and related devices.
\newline To find out the way of using less work sheets.
\newline To establish a system that can many programming languages

\subsection {Research scope}
\subsection {Research significance}
\section{ Methodology}
{ References}
 http://en.wikipedia.org/wiki/snort_(software).
 \newline http://en.wikipedia.org/wiki/metasploit.


\end {document}
